\documentclass[
	12pt,
	a4paper,
	BCOR10mm,
	%chapterprefix,
	DIV14,
	headsepline,
	%twoside,
	%openright
]{scrreprt}

\KOMAoptions{
	listof=totoc,
	bibliography=totoc,
	index=totoc
}

\usepackage[T1]{fontenc}
\usepackage[utf8]{inputenc}

\usepackage{lmodern}

\usepackage[ngerman,english]{babel}

\usepackage[toc]{appendix}
\usepackage{eurosym}
\usepackage{fancyhdr}
\usepackage{graphicx}
\usepackage[htt]{hyphenat}
\usepackage{listings}
\usepackage{microtype}
\usepackage[list=true,hypcap=true]{subcaption}
\usepackage{units}

\usepackage{varioref}
\usepackage[hidelinks]{hyperref}
\usepackage[capitalise,noabbrev]{cleveref}

\lstset{
	basicstyle=\ttfamily,
	frame=single,
	numbers=left,
	language=C,
	breaklines=true,
	breakatwhitespace=true,
	postbreak=\hbox{$\hookrightarrow$ },
	showstringspaces=false,
	tabsize=4,
	captionpos=b,
	morekeywords={gboolean,gpointer,gconstpointer,gchar,guchar,gint,guint,gshort,gushort,glong,gulong,gint8,guint8,gint16,guint16,gint32,guint32,gint64,guint64,gfloat,gdouble,gsize,gssize,goffset,gintptr,guintptr,int8_t,uint8_t,int16_t,uint16_t,int32_t,uint32_t,int64_t,uint64_t,size_t,ssize_t,off_t,intptr_t,uintptr_t,mode_t}
}

\makeatletter
\renewcommand*{\lstlistlistingname}{List of Listings}
\makeatother

\begin{document}

\begin{titlepage}
	\begin{center}
		{\titlefont\huge Adaptive Selection of Lossy Compression Algorithms Using Machine Learning\par}

		\bigskip
		\bigskip

		{\Large Bachelor Thesis\par}

		\bigskip
		\bigskip

		{\large Arbeitsbereich Wissenschaftliches Rechnen\\
		Fachbereich Informatik\\
		Fakultät für Mathematik, Informatik und Naturwissenschaften\\
		Universität Hamburg\par}
	\end{center}

	\vfill

	{\large\begin{tabular}{ll}
		Vorgelegt von:  & Armin Schaare \\
		E-Mail-Adresse: & \href{mailto:3schaare@informatik.uni-hamburg.de}{3schaare@informatik.uni-hamburg.de} \\
		Matrikelnummer: & 6423624 \\
		Studiengang:    & B.Sc. Informatik \\
		\\
		Erstgutachter:  & Julian Kunkel \\
		Zweitgutachter: & Anastasiia Novikova \\ \\
		Betreuer:       & Julian Kunkel \\
		\\
		Hamburg, den 29.09.2016
	\end{tabular}\par}
\end{titlepage}

\chapter*{Abstract}

\thispagestyle{empty}

Lorem ipsum dolor sit amet, consetetur sadipscing elitr, sed diam nonumy eirmod tempor invidunt ut labore et dolore magna aliquyam erat, sed diam voluptua.
At vero eos et accusam et justo duo dolores et ea rebum.
Stet clita kasd gubergren, no sea takimata sanctus est Lorem ipsum dolor sit amet.
Lorem ipsum dolor sit amet, consetetur sadipscing elitr, sed diam nonumy eirmod tempor invidunt ut labore et dolore magna aliquyam erat, sed diam voluptua.
At vero eos et accusam et justo duo dolores et ea rebum.
Stet clita kasd gubergren, no sea takimata sanctus est Lorem ipsum dolor sit amet.

\tableofcontents

\chapter*{Disclaimer}

Chapter \ref{c:intro} and \ref{c:bg} partly consist of the author's same-titled paper, which was written within the scope of the "Abschlussarbeiten-Seminar" of the University of Hamburg.
The seminar was lectured by Dr. Daniel Moldt and the here referenced paper will be made available on-line. \cite{aas} %TODO: link?

\chapter{Introduction}
\label{c:intro}

\section{Motivation}
\label{s:mot}

% HPC unused compression potential
%	- researchers concerned about data quality and speeds
% SCIL being developed
% 	- at dkrz hamburg
%	- part of AIMES project
%	- addresses the problem of increasing data storage efficiency
%	- user interface to common and modern compression methods
%	- user provides error and performance tolerances
%	- scil keeps them and compresses with best possible algorithm at hand
%	-

Compression in HPC contexts still has unused potential.
This is mostly due to researchers concerns about non acceptable data degradation and compression speeds. \cite{ELCC}
For this reason, SCIL - the Scientific Compression Interface Library - for the programming language C is being developed at the German climate computing center in Hamburg.
As part of the AIMES-project, SCIL addresses the problem of increasing data storage efficiency by providing an interface to the most common and modern compression algorithms. \cite{aimes}
The user is able to specify absolute and relative error tolerances as well as compression- and decompression throughputs.
SCIL then automatically decides which algorithm is best suited for the given arguments and data to compress. \cite{scil}
The decision process is the main subject of this thesis. \par

As there is a whole spectrum of different compression algorithms, choosing the optimal one for the given data is quite involved.
The user has to keep in mind the overall structure of the data such as its minimum and maximum value, standard deviation, mean value, etc.
Furthermore, an in depth knowledge of each compression method, its strengths and weaknesses and on what data of which structure it performs the best, is required. % TODO: Satz ist komisch
Therefore, users tend to have only a few lossless and lossy compression methods in their repertoire, sticking to one of them for every compression.
This leads to cases where the compression with a completely different method - unbeknown to the user - would be beneficial regarding compression speed, ratio or both.
SCIL aims to completely abstract the decision process from the user, by applying machine learning to map relevant criteria to the best suitable method.
%With this it is hoped, that SCIL will be able to optimally compress data in the overwhelming majority of tasks.

\section{Goal of the Thesis}
\label{s:gott}

% TODO: ausarbeiten

The goal of this thesis is to evaluate different machine learning models for the task of SCIL's automatic compression decider.
Models under evaluation are Logistic Regression, Decision Trees, Feedforward Neural Networks and for comparison purposes the Mean Value approach.
After the conclusion of this paper it should be clear, which one of these approaches will be best suitable to model the automatic decision process.

\section{Structure of the Thesis}
\label{s:sott}

This thesis begins by providing a detailed background of all involved fields and how the automatic compression decider depends on these.
An explanation of SCIL's inner functioning follows, shining light on the surrounding structure and possible difficulties of including the decision module. % TODO: hört sich komisch an
The background chapter finishes by referencing related works and further reading. \par % TODO: Eigenes Chapter?

In chapter \ref{c:des} the design of the compression decider and its training process is portrayed.
It is explained, which parameters could be important for training the machine learning model, how compressible data is generated and how further relevant characteristics are extracted from the resulting data. % TODO: komisch
The following implementation chapter addresses the realization of the theoretical foundation provided by the design chapter.
In regard to the generation of data (either compressible or training data), C code is provided with short explanations of more obscure passages.
The training and evaluation of machine learning tasks is realized with the statistical programming language R.
For this, a short assessment of R, its capabilities and limits regarding machine learning is conducted.
Chapter \ref{c:eval} addresses the evaluation of the considered machine learning models.
The iterative approach of including compression relevant parameters in the training offers a broad field of possible analysis.
While most significant iterative steps and their evaluation graphs are available in the chapter itself, graphs and evaluations of each single steps are listed in the appendix. % TODO: wird das tatsächlich so sein?
The summary and conclusion chapter will present the analysis of the evaluation's outcome, answering the leading question which machine learning approach is best suited for the task at hand.
Finally the thesis finishes with future outlooks of machine learning in conjunction to data compression.

\chapter{Background} % 10 bzw. 12 Seiten
\label{c:bg}

\section{Compression}
\label{s:comp}

% - is data encoding using less bits
% - lossy and lossless
% - lossless: always precise, lossy: tradeoff between precision and performance
% - lossless has limit
% - lossy has no limit, but information will be degraded a lot

% - history

% - examples of compression today

% - minimizes storage and bandwidth use
% - Therefore applications everywhere
% - widely used (audio, video, internet, hpc)
% - still estimate that worldwide data can be further compressed by factor of 4.5 lossless
% - with lossy, a lot more

% - critics

% - compression is overhead
% - fear of unsuitable lossy precisions

% - outlook of compression

Compression is the procedure of encoding data using less memory space than its uncoded equivalent.
It can be separated into two categories: lossy and lossless compression.
While lossless compression has to retain every bit of information, lossy compression represents a trade-off between data degradation and size.
For this reason, lossy algorithms generally produce better ratios of uncompressed and compressed data sizes than lossless ones.
The main applications of data compression are minimizing storage and bandwidth use.
Therefore it is widely used in digital audio, video and text as well as commercial and scientific data processing. \par

First applications of data compression were present as early as 1838 with the invention of Morse code.
For this, common letters of the English alphabet were encoded by shorter strings of bits (or in this context short signals versus long ones). % TODO: komisch mit der Klammer
Without use cases in that time, further notable advances of data compression where delayed to the 1940s and the rise of information theory.
1949, Claude Shannon and Robert M. Fano devised first methods of compression by using statistical approaches.
This lead to David A. Huffman - at that time a Ph.D. student under Robert M. Fano - perfecting the scheme two years later, which is now known as the famous and widely used \textit{Huffman coding}. \cite{huffenc}
In the following years, Huffman coding was implemented directly on hardware before being enhanced by dynamically coding actual encountered data instead of maintaining a static map.
The dynamic mapping, developed in the 1970s, was widely used due to the rise of the on-line storing of text files.
Shortly after, a fundamentally different algorithm emerged through the work of Abraham Lempel and Jacob Ziv, firstly realized by Terry A. Welch in 1984. \cite{welchenc}
The \textit{Lempel-Ziv-Welch} algorithm (LZW) quickly became the de facto standard due to unprecedented compression ratios and is still widely used today. \cite{llcomp}
With personal computers capability of storing ever increasing amounts of data, digital images and audio became more popular. \cite{anks}
First lossy compression methods were developed for commercial and private use, most notably JPEG for digital images and MP3 for audio files.
Since a degradation of picture or sound quality does not necessarily devalue the data in a way, errors would impact text files or scientific data, it can be reasonably traded-off for smaller file sizes.
Both file formats were released in the early 1990s and are still widely in use today, even if there are many alternatives performing objectively better (i.e JPEG's successor JPEG 2000). \cite{jpeg}
In the case of scientific data, lossy compression is frowned upon, because it implies loss of information.
Even though institutions like NASA encouraged use of lossy compression for research data, an adoption was nowhere to be seen. \cite{nasa}
Science kept relying on lossless methods as of now. \par


%As a result the JPEG method was released in the early 1990s, with compression ratios of 15:1 for noticeable but - depending on the usage - justifiable loss of image quality. \cite{jpeg}
%The audio equivalent of the JPEG file format certainly would be MP3.
%It was released only one year after JPEG in 1993 and reaches compression ratios of 11:1 on a commonly used encoding setting, relative to CD-quality files. \cite{mp3}
%These two methods have prevailed while standing out in comparison to other lossy algorithms.
%Remarkably, the objectively better JPEG 2000 was developed to succeed JPEG, but never gained track.
%Lossy alternatives to MP3 also exist, though it is questionable whether they are truly better than MP3 regarding sound quality or rather based on different individual perception.
%In the early 2000s, computers became powerful enough to allow for a video surge through the digital world.
%Quickly, the file formats MP4, AVI, and MPEG - to name a few - were created.
%They relied on new video encodings as well as already available audio and even text compression for subtitles.


Due to compression minimizing storage and bandwidth needs, it has many applications and is already widely used.
The Internet for example makes use of lossless compression in between servers and clients to provide error-free (syntax- and content-wise) web pages.
Generally, files containing text are always compressed using lossless algorithms, since inaccuracies in such cases are not justifiable.
Lossy compressors on the other hand are

\section{Machine Learning}
\label{s:ml}

\section{Data Generation}
\label{s:dg}

\section{SCIL}
\label{s:scil}

\section{Related Work} % 2 Seiten, TODO: Eigenes Chapter?
\label{s:rw}

\chapter{Design}
\label{c:des}

\chapter{Implementation}
\label{c:impl}

\chapter{Evaluation}
\label{c:eval}

\chapter{Summary and Conclusions}
\label{c:sac}

%\chapter{Introduction}
%\label{Introduction}
%
%\textit{%
%In this chapter, ...
%}
%
%\bigskip
%
%Lorem ipsum dolor sit amet, consetetur sadipscing elitr, sed diam nonumy eirmod tempor invidunt ut labore et dolore magna aliquyam erat, sed diam voluptua.
%At vero eos et accusam et justo duo dolores et ea rebum.
%Stet clita kasd gubergren, no sea takimata sanctus est Lorem ipsum dolor sit amet.
%Lorem ipsum dolor sit amet, consetetur sadipscing elitr, sed diam nonumy eirmod tempor invidunt ut labore et dolore magna aliquyam erat, sed diam voluptua.
%At vero eos et accusam et justo duo dolores et ea rebum.
%Stet clita kasd gubergren, no sea takimata sanctus est Lorem ipsum dolor sit amet.\footnote{I am a footnote.}
%
%Lorem ipsum dolor sit amet, consetetur sadipscing elitr, sed diam nonumy eirmod tempor invidunt ut labore et dolore magna aliquyam erat, sed diam voluptua.
%At vero eos et accusam et justo duo dolores et ea rebum.
%Stet clita kasd gubergren, no sea takimata sanctus est Lorem ipsum dolor sit amet:
%
%\begin{itemize}
%	\item Lorem
%	\item Ipsum
%\end{itemize}
%
%Lorem ipsum dolor sit amet, consetetur sadipscing elitr, sed diam nonumy eirmod tempor invidunt ut labore et dolore magna aliquyam erat, sed diam voluptua.
%At vero eos et accusam et justo duo dolores et ea rebum.
%“Stet clita kasd gubergren, no sea takimata sanctus est Lorem ipsum dolor sit amet.”~\cite{Quelle2012}
%
%\medskip
%
%Reference example: For more information, see
%\begin{itemize}
%	\item Chapter~\ref{Assignment},
%	\item \cref{Assignment},
%	\item \vref{Assignment}.
%\end{itemize}
%
%\chapter{Assignment}
%\label{Assignment}
%
%\textit{%
%In this chapter, ...
%}
%
%\bigskip
%
%\section{Section}
%
%Lorem ipsum dolor sit amet, consetetur sadipscing elitr, sed diam nonumy eirmod tempor invidunt ut labore et dolore magna aliquyam erat, sed diam voluptua.
%At vero eos et accusam et justo duo dolores et ea rebum.
%Stet clita kasd gubergren, no sea takimata sanctus est Lorem ipsum dolor sit amet.
%Lorem ipsum dolor sit amet, consetetur sadipscing elitr, sed diam nonumy eirmod tempor invidunt ut labore et dolore magna aliquyam erat, sed diam voluptua.
%At vero eos et accusam et justo duo dolores et ea rebum.
%Stet clita kasd gubergren, no sea takimata sanctus est Lorem ipsum dolor sit amet (see \cref{fig:logo}).
%
%\begin{figure}[h]
%	\centering
%	\includegraphics[height=0.2\textheight]{logo.jpg}
%	\caption{Example figure}
%	\label{fig:logo}
%\end{figure}
%
%Lorem ipsum dolor sit amet, consetetur sadipscing elitr, sed diam nonumy eirmod tempor invidunt ut labore et dolore magna aliquyam erat, sed diam voluptua.
%At vero eos et accusam et justo duo dolores et ea rebum.
%Stet clita kasd gubergren, no sea takimata sanctus est Lorem ipsum dolor sit amet.
%Lorem ipsum dolor sit amet, consetetur sadipscing elitr, sed diam nonumy eirmod tempor invidunt ut labore et dolore magna aliquyam erat, sed diam voluptua (see \cref{table:table}).
%
%\begin{table}
%	\centering
%	\begin{tabular}{|c|c|c|}\hline
%		Column 1 & Column 2 & Column 3 \\
%		\hline
%		Row 1    &   -      & -        \\
%		Row 2    &  Row 2   & -        \\
%		Row 3    &  Row 3   & Row 3    \\
%		\hline
%	\end{tabular}
%	\caption{Example table}
%	\label{table:table}
%\end{table}
%
%\chapter{Design}
%\label{Design}
%
%\textit{%
%In this chapter, ...
%}
%
%\bigskip
%
%Lorem ipsum dolor sit amet, consetetur sadipscing elitr, sed diam nonumy eirmod tempor invidunt ut labore et dolore magna aliquyam erat, sed diam voluptua.
%At vero eos et accusam et justo duo dolores et ea rebum.
%Stet clita kasd gubergren, no sea takimata sanctus est Lorem ipsum dolor sit amet (see \cref{lst:listing}).
%
%\begin{lstlisting}[caption=Example listing, label={lst:listing}]
%void print (void)
%{
%	int i;
%
%	for (i = 0; i < 10; i++)
%	{
%		printf("Hello World!\n");
%	}
%}
%\end{lstlisting}
%
%\chapter{Conclusion}
%\label{Conclusion}
%
%Lorem ipsum dolor sit amet, consetetur sadipscing elitr, sed diam nonumy eirmod tempor invidunt ut labore et dolore magna aliquyam erat, sed diam voluptua.
%At vero eos et accusam et justo duo dolores et ea rebum.
%Stet clita kasd gubergren, no sea takimata sanctus est Lorem ipsum dolor sit amet.

\bibliographystyle{alpha}
\bibliography{literatur}

\appendix
\appendixpage

\chapter{Chapter}

Lorem ipsum dolor sit amet, consetetur sadipscing elitr, sed diam nonumy eirmod tempor invidunt ut labore et dolore magna aliquyam erat, sed diam voluptua.
At vero eos et accusam et justo duo dolores et ea rebum.
Stet clita kasd gubergren, no sea takimata sanctus est Lorem ipsum dolor sit amet.

\listoffigures

\lstlistoflistings

\listoftables

\chapter*{}

\thispagestyle{empty}

\section*{Eidesstattliche Versicherung}

\begin{otherlanguage}{ngerman}
Hiermit versichere ich an Eides statt, dass ich die vorliegende Arbeit im Studiengang XXX selbstständig verfasst und keine anderen als die angegebenen Hilfsmittel -- insbesondere keine im Quellenverzeichnis nicht benannten Internet-Quellen -- benutzt habe.
Alle Stellen, die wörtlich oder sinngemäß aus Veröffentlichungen entnommen wurden, sind als solche kenntlich gemacht.
Ich versichere weiterhin, dass ich die Arbeit vorher nicht in einem anderen Prüfungsverfahren eingereicht habe und die eingereichte schriftliche Fassung der auf dem elektronischen Speichermedium entspricht.

\bigskip

\noindent
Ich bin damit einverstanden, dass meine Abschlussarbeit in den Bestand der Fachbereichsbibliothek eingestellt wird.
\end{otherlanguage}

\bigskip
\bigskip
\bigskip

\begin{center}
\begin{tabular}{ll}
	\rule{0.35\textwidth}{0.4pt} & \rule{0.55\textwidth}{0.4pt} \\
	Ort, Datum & Unterschrift
\end{tabular}
\end{center}

\end{document}
