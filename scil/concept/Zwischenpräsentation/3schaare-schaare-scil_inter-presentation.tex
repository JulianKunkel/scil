\documentclass[compress]{beamer}

\usetheme{Hamburg}

\usepackage[T1]{fontenc}
\usepackage[utf8]{inputenc}

\usepackage{lmodern}

%\usepackage[english]{babel}
\usepackage[ngerman]{babel}

\usepackage{eurosym}
\usepackage{listings}
\usepackage{microtype}
\usepackage{units}

\lstset{
	basicstyle=\ttfamily\footnotesize,
	frame=single,
	numbers=left,
	language=C,
	breaklines=true,
	breakatwhitespace=true,
	postbreak=\hbox{$\hookrightarrow$ },
	showstringspaces=false,
	tabsize=4,
	captionpos=b,
	morekeywords={gboolean,gpointer,gconstpointer,gchar,guchar,gint,guint,gshort,gushort,glong,gulong,gint8,guint8,gint16,guint16,gint32,guint32,gint64,guint64,gfloat,gdouble,gsize,gssize,goffset,gintptr,guintptr,int8_t,uint8_t,int16_t,uint16_t,int32_t,uint32_t,int64_t,uint64_t,size_t,ssize_t,off_t,intptr_t,uintptr_t,mode_t}
}

\title{SCIL - Scientific Compression Interface Library}
\author{Armin Schaare}
\institute{Arbeitsbereich Wissenschaftliches Rechnen\\Fachbereich Informatik\\Fakultät für Mathematik, Informatik und Naturwissenschaften\\Universität Hamburg}
\date{2016-01-17}

\titlegraphic{\includegraphics[width=0.75\textwidth]{logo}}

\begin{document}

\begin{frame}
	\titlepage
\end{frame}

\begin{frame}
	\frametitle{Outline}

	\tableofcontents[hidesubsections]
\end{frame}

\section{Introduction}
\subsection*{}

\begin{frame}
	\frametitle{Introduction}

	What is SCIL?
	\bigskip
	\begin{itemize}
		\item C-Library for compression
		\item Offers many different  compression algorithms
		\item User provides arguments for lossy compression
		\item SCIL chooses which algorithm is best suited
		\item User can force specific compression algorithms
	\end{itemize}

\end{frame}

\begin{frame}
	\frametitle{Usage}

	\begin{itemize}
		\item Define lossy arguments in 'scil\_hints' struct
		\item Create compression context
		\item Allocate destination buffer
		\item Use context to compress the given buffer
	\end{itemize}

\end{frame}

\begin{frame}[fragile]
	\frametitle{Usage Example Code}

	\begin{lstlisting}[caption=SCIL usage example]
	scil_hints hints;
	hints.force_compression_method = 0;

	scil_context * ctx;
	scil_create_compression_context(&ctx, &hints);

	size_t c_size;
	scil_compress(ctx, c_buf, &c_size, u_buf, u_size);
	\end{lstlisting}

	\bigskip

	\begin{tabular}{ll}
		c\_buf: & allocated destination buffer \\
		c\_size: & will be byte size of c\_buf \\
		u\_buf: & uncompressed source buffer \\
		u\_size: & byte size of u\_buf
	\end{tabular}

\end{frame}

\section{Hauptteil}
\subsection*{}

\begin{frame}[fragile]
	\frametitle{Hauptteil}

	\begin{itemize}
		\item Hauptpunkt 1
		\begin{itemize}
			\item Unterpunkt 1
			\item Unterpunkt 2
		\end{itemize}

		\begin{lstlisting}[caption=Beispielquelltext]
		int foo (void)
		{
		    return 0;
		}
		\end{lstlisting}
	\end{itemize}
\end{frame}

\section{Zusammenfassung}
\subsection*{}

\begin{frame}
	\frametitle{Zusammenfassung}

	\begin{itemize}
		\item Zusammenfassung 1
		\begin{itemize}
			\item Unterpunkt 1
			\item Unterpunkt 2
		\end{itemize}
		\item Zusammenfassung 2
		\begin{itemize}
			\item Unterpunkt 1
			\item Unterpunkt 2
		\end{itemize}
		\item Quelle: \cite{Quelle2012}
	\end{itemize}
\end{frame}

\end{document}
