\documentclass[compress]{beamer}

\usetheme{Hamburg}

\usepackage[T1]{fontenc}
\usepackage[utf8]{inputenc}

\usepackage{lmodern}

%\usepackage[english]{babel}
\usepackage[ngerman]{babel}

\usepackage{eurosym}
\usepackage{listings}
\usepackage{microtype}
\usepackage{units}

\lstset{
	basicstyle=\ttfamily\footnotesize,
	frame=single,
	numbers=left,
	language=C,
	breaklines=true,
	breakatwhitespace=true,
	postbreak=\hbox{$\hookrightarrow$ },
	showstringspaces=false,
	tabsize=4,
	captionpos=b,
	morekeywords={gboolean,gpointer,gconstpointer,gchar,guchar,gint,guint,gshort,gushort,glong,gulong,gint8,guint8,gint16,guint16,gint32,guint32,gint64,guint64,gfloat,gdouble,gsize,gssize,goffset,gintptr,guintptr,int8_t,uint8_t,int16_t,uint16_t,int32_t,uint32_t,int64_t,uint64_t,size_t,ssize_t,off_t,intptr_t,uintptr_t,mode_t}
}

\title{Titel des Vortrags}
\author{Name des Autors}
\institute{Arbeitsbereich Wissenschaftliches Rechnen\\Fachbereich Informatik\\Fakultät für Mathematik, Informatik und Naturwissenschaften\\Universität Hamburg}
\date{2015-01-01}

\titlegraphic{\includegraphics[width=0.75\textwidth]{logo}}

\begin{document}

\begin{frame}
	\titlepage
\end{frame}

\begin{frame}
	\frametitle{Gliederung (Agenda)}

	\tableofcontents[hidesubsections]
\end{frame}

\section{Einleitung}
\subsection{Untersektion}

\begin{frame}
	\frametitle{Einleitung}

	\begin{itemize}
		\item Hauptpunkt 1
		\begin{itemize}
			\item Unterpunkt 1
			\item Unterpunkt 2
		\end{itemize}
	\end{itemize}

	\begin{figure}
		\begin{center}
			\includegraphics[width=0.75\textwidth]{logo.jpg}
		\end{center}
		\caption{Beispielgrafik}
	\end{figure}
\end{frame}

\section{Hauptteil}
\subsection{Untersektion}

\begin{frame}[fragile]
	\frametitle{Hauptteil}

	\begin{itemize}
		\item Hauptpunkt 1
		\begin{itemize}
			\item Unterpunkt 1
			\item Unterpunkt 2
		\end{itemize}

		\begin{lstlisting}[caption=Beispielquelltext]
		int foo (void)
		{
		    return 0;
		}
		\end{lstlisting}
	\end{itemize}
\end{frame}

\section{Zusammenfassung}
\subsection*{}

\begin{frame}
	\frametitle{Zusammenfassung}

	\begin{itemize}
		\item Zusammenfassung 1
		\begin{itemize}
			\item Unterpunkt 1
			\item Unterpunkt 2
		\end{itemize}
		\item Zusammenfassung 2
		\begin{itemize}
			\item Unterpunkt 1
			\item Unterpunkt 2
		\end{itemize}
		\item Quelle: \cite{Quelle2012}
	\end{itemize}
\end{frame}

\section{Literatur}
\subsection*{}

\begin{frame}
	\frametitle{Literatur}

	\bibliographystyle{alpha}
	\bibliography{literatur}
\end{frame}

\end{document}
